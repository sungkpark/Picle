\chapter{Value Sensitive Design}\label{ch:ethics}
Programs have an impact on people. But not only on the people who directly use said programs.
There are also other stakeholders. In this chapter we will elaborate on how the design
and implementation of Picle would have changed, if we would have looked at more than only
the users as stakeholders of Picle. The other stakeholder group we will discuss will be
that of (wild) animals.

This stakeholder analysis is important, because we, as programmers and computer scientists,
have a moral obligation to take this into account. The specialization of what people do
that started with the agricultural revolution and has increased ever since, means that
a small group of programmers decide what the non-programmer population can do with their
computers. This is a great power, and thus comes with great responsibilities of which
we should be all aware.

If we would have focused more on these two stakeholder groups, gamification would have
been a bigger part of the final product. This because this drives competition between
users, which will in turn help the users be more engaged and have more fun and the
users will make their lifestyle more ecological. Which in turn will increase animal
welfare.

The two stakeholder groups are thus \emph{users} and \emph{(wild) animals}. The users of
our application are an easily definable group: it is just everyone who installs the client
and uses Picle. The other group, the (wild) animals, are harder to define. Because how
does our application have an effect on this group? The users of our application have an
impact on the animals around them: the wild animals that live in their region, but also
the animals they eat and those that live in places endangered by climate change or
pollution. If Picle encourages our users to live a more earth friendly life, they also
improve the lives of the animals they affect.

We can easily see that the interests of both stakeholder groups overlap a lot: they both
benefit from a more earth friendly lifestyle. However, there is some tension between
both groups. For example, if we would introduce a penalty for eating meat or consuming
animal products, that would be better for the animal welfare, but not all of our users
would probably like to change to a completely vegan diet.

If we focus on the overlapping interests, the most important thing would be to make Picle
more accurate and to add more possible activities. For this, there is still a lot of room
for improvement. We are obviously no environmental scientists, so that is a group of
people we would need a lot of advice for. The more accurate Picle is, the better it can
help people with an earth friendly lifestyle.

For adding more activities, the best way would be to crowd source ideas from our users.
We are with just seven people, so the diversity of the ideas from our users will definitely
be greater than what we ourselves could come up with. This would however require a
redesign of the GUI.
