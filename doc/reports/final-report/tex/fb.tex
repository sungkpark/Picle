\chapter{Individual feedback}\label{ch:individual-feedback}

\section{Andreas Achilleos}
Kickstarting the project proved rather difficult. Once we had built a proper infrastructure 
everything went by a lot quicker. Each member was assigned on a different part of the project development while 
keeping our client – server team division. 

I worked almost exclusively on the User Interface; designed the initial wireframes and then implemented the scenes.
I used CSS files for styling and custom libraries for extra functionality.
I designed and animated our logo and when the basic scenes were built, the client team members wrote controllers to 
deal with user input and error handling. Then I connected everything together to create the finalized interface.

Getting accustomed with the new tools was the most difficult part of the development for me. Having no prior 
experience in team-developing and large scale projects, learning how to use Git was a crucial tasks. 
I had to spent many hours on research so I could get on the same level as some of my more experienced teammates. 

To compensate for my weaknesses, I was present in all group meetings and met all deadlines both from our team 
and the course schedule. Some features took longer to implement and the more experienced team members had to work
extra to fix errors that I was generating. Getting feedback from different members helped me to personally understand 
complex topics and generate a result that we were all happy with.

Overall I am happy with our team performance and final product. No real problems were raised that 
we couldn't resolve within our team.
  

\section{Alp Capanoglu}
I am satisfied with my contribution to the project, and I surpassed the expectations I set upon myself.
I was able to work productively and feel that I have communicated meaningfully with my teammates.

I feel like my strongest part was how I was able to learn a lot in the first weeks to contribute
meaningfully to the project. I met all goals on time, wrote many of the tests and most of server-side
authentication as well as some work on the other layers of the server. I am relieved and proud, because
my biggest doubt of myself was that I would not learn enough and underperform.

Even though I have met all my deadlines, I feel I used my time inefficiently, as my time working on the
project was intermittent and usually late. This was inefficient in every other sense than producing code.
Now that I realize that this is bad practice, I aim to improve my time management skills even further.

The most difficulty I had in the project was tackling the steep learning curve (tooling, testing, filters
and beans). I overcame this through consistent and hard work. I also asked friends, primarily Martijn and Sung,
when I needed help. Ironically, this was perhaps when I best managed my time (on a personal scale). The end result
of passing this hurdle, however, has been the large amount of new skills I have developed and meaningful contributions
I have made to the project.


\section{Jordy Koemans}
For this project I mostly worked on the client-side of our team. At first I was going to be part of coding the features 
but this didn't go so well. I was having difficulty with getting started and keeping up with the speed of the project. 
It was pointed out that I wasn't clear on what I was working on and what I was having trouble with which I agreed with. 
Clear communication in group projects is something that I have difficulty with and was something I wanted to work on. After 
that was pointed out I tried focussing on it more which I think I did improve on.

After that I started working on researching the activities which was able to do and it helped me contribute more to the project
which had been a problem for me and the group up to that point. Once that was done I started of increasing test coverage which 
proved to be a difficult task for me but thanks to a lot of help from the team members I was able to do that successfully.

In the end I did improve on the points I wanted to work on but not as much as I'd have liked. I lost track of the project from 
the start and my communication wasn't clear enough, the last few weeks went a lot better in that regard and it did help me a lot. 
I would like to get better at coding in Java so I can contribute more in following projects but looking at the work my team members 
made did teach me a lot.

\section{Omar Sheasha}
Throughout this project, I have been working on the client-side aspect of PICLE. 
I have learnt how to coherently cooperate in a team and take on tasks that I did not have experience in. I learnt how to manage my time and adapt my schedule to always meet the deadline for the demo, and keep the team’s workflow steady without holding them back. I also mostly learnt how all the different components play a role in reaching our goal.
I have been able to take on the responsibility of being a chair of our weekly meetings as well as taking minutes in our meetings. Being the chair of the meeting required me to be able to identify our upcoming challenges, and clearly present them to the team so that we can decide upon how to tackle them and proceed onwards. I believe I was able to present myself well, make those points clear and take everyone’s opinion on how to tackle the discussed issues.
I also learnt to take criticism or feedback and act on it as well as not shunning away from reaching out for help when I am faced with something I can not resolve.
I believe I have acted upon all the points of my personal development plan.

\section{Sung Park}
My main contribution was structuring the server side by setting up a N-tier layer where each layer has a smaller and definitive scope of function.
This structure organizes the server-side with modularity, and enables testing much clearer.
Implementing the functionality of activities, users, and scores with tests were part of my job as well.

My strength was finishing assigned works on time without any trouble.
I would know what I would have to do before the next meeting and I would have those tasks done before the next meeting.
Since my tasks were mostly implementing server side implementations, I would also have to keep the test coverage up to par, which I did not let down on before a readied merge request.

One of my weaker point was communication.
At the beginning I worked on the structure of the server-side, which needs communication to our team, where I saw my communication skill slacking off.
I found my group left unclear, however with documentation added about the structure, WhatsApp, GitLab (especially code review and discussion), I saw our whole team communicating very well including myself.

One small problem I had was not reviewing every code thoroughly.
Once, I had implemented a method which was already implemented.
This was because I hadn't reviewed the merge request which implemented that method.
It caused a bit of confusion amongst the server-side people, but through communication and code review from GitLab this was solved and pointed out easily.

\section{Martijn Staal}
I am very satisfied with what I contributed to the project, on all parts: code, organisation,
planning and with the report. I also think that I helped my team members with their tasks
and with with learning things I am more experienced with (mostly git, LaTeX). Although I
think that my direct way of communication might have seemed a bit harsh in the first weeks.
But I think that all in all I can positively say that I have improved my collaboration
skills, as I intended in my Personal Development Plan.

My weaker point during this project was that I didn't do that much research into the
framework we used for building the application. I mostly asked Alp how everything worked
and used the existing code in the project as examples. This however does show that we
were able to effectively work in a team.

I think that there were little problems during the project in our group. Most of the time
we were able to find consensus on the issues we faced, which were mostly design choices.
Sometimes it took more than one meeting to make a decision, but I think that this was
necessary. In the end, this gave us the extra time we needed to make a final decision.

\section{Just van der Veeken}
Personally I think the project was a successful team effort in which every team member was satisfied with their assigned tasks and the final product while maintaining a balanced workload. Also, it really thought me how to properly research technology I did not know.

For this project I feel like my contribution was satisfactory. I think I can safely say I have at least some knowledge about every aspect of the project.
This put me in a position where I was able to help out my team members often, which matches the ambitions I had at the start of the project.
Besides that I tried keeping an eye on how everyone was feeling about the work they were doing. As a result some of us shifted their main focus for the project so that they experienced it to be more pleasant or they felt that they were contributing more that way.

The most difficult aspect of the project was time management. A lack of proper planning resulted in me learning it the hard way. Namely, working all night long to pass one of the deadlines.
Complementary to planning, using your time efficiently is a whole different skill which I could have improved on. For some tasks I spent much time struggling with the implementation while doing some more extensive research in advance could have saved me a lot of time.

In conclusion, I think the project thought me some valuable lessons on project management and team work with proper time management being the hardest nut to crack.
