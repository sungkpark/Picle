\chapter{Reflection}\label{ch:reflection}

\section{General Overview}
The course provided the groups with a unique experience of a real world developing scenario. 
Our result is definitely not perfect but each member is now familiar with the process of
creating an actual product.

\section{Group Management}
Our group was split into two teams working on client/server individually.
We updated our backlog regularly and assigned individual tasks while tracking the progress through 
WIP merge requests. Our planning was quite often “too optimistic” so we should have set more strict goals 
for each sprint aiming for a more stable timetable. We had a weekly brief on every meeting with our TA present. 
Updating the team during the week could have provided more insight on the progress and/or signal 
the rest members if the workload needs to be shifted.

\section{Developing Picle}
The product covers all basic features listed in the course’s rubric, 
but we could always expand and add more features if it was possible in the time frame.

Improvements for the client would include, more configuration options
and additional gamification aspects that provide even more stimuli to the user.

For the sake of simplicity and a user-friendly design we had to limit our activity parameters sacrificing score accuracy.
Additional options for activities would greatly increase the accuracy of carbon savings
and provide more detailed information.

\section{Following the OOP Project}
The structure of the course helped both experienced and inexperienced team members to
get familiarized with a more professional environment. 
To recreate a customer-developer relationship the groups were mostly left in the dark 
on any technical matters, so implementing new features proved rather slow in the beginning. 
An improvement from the course staff would be to provide basic information on a more complete toolset to kickstart a project.
