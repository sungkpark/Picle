\chapter{Product}\label{ch:product}

\section{Description and goal}
Our product is called \emph{Picle}, which is an acronym for ``People Interactively Creating
a Livable Environment''. Picle aims to enable users to make their ecological footprint
smaller by tracking the saved amount of CO\textsubscript{2}-equivalents, comparing them
with their friends and rewarding them achievements.

\section{Product structure}
The product consists of a client, an application server and a database server.

The application server and the client communicate with each other via HTTP Basic Authenticated
requests in a RESTful API. This API is defined in the \verb+doc/+ folder of this project.

Picle is focused around \emph{activities}. Any thing that a user can do that has an impact
on their CO\textsubscript{2} emissions could be made into an activity. We however needed
to make a compromise to keep the score calculation maintainable and user friendly. For
more accuracy you need more parameters, but the user obviously wants to fill in the least
possible things.

Activities are added to their personal database of activities by the user. Each activity
modifies the score, which is the amount of kilograms of saved CO\textsubscript{2}-equivalent
emissions saved, of the user. For each activity, we've established a formula to calculate
this score with either one or two integer parameters. These formulas are obviously an
approximation of reality, but with our research we have confidence that it portrays
reality accurately enough for the goals of our application.

\subsection{Client}
The client is ``thin'', which is to say that the client doesn't store or calculate anything.
This is to protect the integrity of the data and scores so the attack surface for
possible cheaters is minimized. The thin nature of the client also enables extreme
portability: users can log in on any device and still have all their data ready. The
downside of this, is that a lot of data has to be fetched from the server when a new
client connect. Another possible issue is that a user is not able to view their activities
and score when their client is not connected to the server.

The client has a graphical user interface, which is made using JavaFX.\footnote{\url{https://openjfx.io/}}

\subsection{Application server}
The application server is a server written in Java 11 using the Spring Boot framework.\footnote{\url{https://spring.io/projects/spring-boot}}
The application server is the computational brain of the whole Picle application.

When a user adds an activity, the client will send a \verb+POST+ request to the server.
The server then first checks if the user is authenticated correctly. If it is authenticated
correctly, the server calculates the score and stores the activity in the database.

Most other workflow on the server works in a comparable way: an authenticated HTTP request
is sent by the client which is then processed accordingly by the application server.

This server centric approach makes it easy to have full control over the Picle application,
but it has some downsides. Scaling the Picle application for a big amount of users becomes
relatively hard with this approach, certainly when a client always retrieves all their
data when a user starts the client.

\subsection{Database server}
For the database server, we have chosen to use MariaDB.\footnote{\url{https://mariadb.org/}}
We chose MariaDB because it is a simple, free and open source SQL server.

\section{Security}
While designing and building our product we have mostly looked at three facets of security:
data integrity, authentication and confidentiality of user data.

\subsection{Data integrity}
Since Picle is also partially a game, with the competitive aspects of comparing scores,
it has to be a fair game. Therefore, we need to make sure that cheaters have no chance.
This is done by making the server trust the client for absolutely the minimum amount
necessary. Everything that can be done on the server, is done on the server.

\subsection{Authentication}
Every individual request from the client to the server is authenticated using HTTP Basic
Authentication. With this approach, there is no such thing as a 'session', so any
session stealing kind of attack is impossible. The passwords are stored hashed and salted
on the database using PBKDF2.

\subsection{Confidentiality of user data}
Our application server is designed and implemented in such a way that a user can only
access their own activities. This is important, because a user's activities may say a lot
about their habits, if they use Picle a lot.
